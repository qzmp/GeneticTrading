%\documentclass[twoside]{pwrthesis}
\documentclass[twoside]{iisthesis}
% --
\usepackage{polski}
\usepackage[cp1250]{inputenc}
\usepackage{graphicx}

% Dodane przeze mnie d
\usepackage{fancyvrb} % dla srodowiska Verbatim
\usepackage{color}
\usepackage{lscape}


% definicje kolorow
\definecolor{ciemnoSzary}{rgb}{0.15,0.15,0.15}
\definecolor{szary}{rgb}{0.5,0.5,0.5}
\definecolor{jasnoSzary}{rgb}{0.2,0.2,0.2}

% Konfiguracja verbatima
\fvset{
	frame=single,
	numbers=left,
	fontsize=\footnotesize,
	numbersep=12pt,
%	framerule=.5mm,
	rulecolor=\color{ciemnoSzary},
%	fillcolor=\color{jasnoSzary},
	framesep=4pt,
	stepnumber=1,
	numberblanklines=false,
	tabsize=2,
%	formatcom=\color{szary}
}

\usepackage{wrapfig}

\begin{document}


\chapter*{Wprowadzenie}

\section{Czym jest metaheurystyka}
Wyjaśnienie czym jest metaheurystyka wymaga najpierw opisanie czym jest sama heurystyka. Heurystyka jest szczególnym rodzajem algorytmu, który nie gwarantuje otrzymania optymalnych rozwiązań. Wydawałoby się to wadą, jednak wiele problemów nie wymaga podania optymalnego rozwiązania. Heurystyka wykorzystuje fakt, że w określonych przypadkach jesteśmy gotowi tolerować pewną niepewność. Grupą problemów, w których często wykorzystywane są heurystyki są zadania optymalizacji. Na przykład, wybierając najkrótszą drogę do sklepu, nie liczymy dokładnie ile czasu nam zajmie każda możliwość. Wystarczy nam jedynie oszacowanie i to zapewnia nam nasz mózg niby automatycznie, jednak często po drodze wykonując skomplikowane obliczenia. Bierzemy pod uwagę wiele czynników, nie tylko długość drogi, ale także ruch samochodów, stan nawierzchni czy subiektywne wrażenia. Nasz wybór może nie być idealny, ale takie podejście, właśnie heurystyczne, ma wiele zalet. Gdybyśmy dokładnie obliczali wszystkie możliwości, tracilibyśmy na to wiele energii, która dopiero w ostatnim okresie historycznym jest szeroko dostępna. Co więcej, możliwości obliczeniowe naszych mózgów są ograniczone, więc rozwiązywanie tego prostego przecież problemu zajęłoby nam wiele czasu.

Mimo, że komputery mają znacznie większe możliwości przetwarzania danych niż człowiek, niektórych problemów nie da się po prostu przeanalizować w całości. W niektórych przypadkach przestrzeń rozwiązań jest tak duża, że sprawdzenie wszystkich możliwości rozwiązań może zająć zbyt dużo czasu. Takie podejście, zwane siłowym, (lepiej znane jest określenie z języka angielskiego ``''brute force'') jest proste w implementacji i wydaje się być naturalne dla komputerów.

Pierwsze naukowe próby heurystycznego rozwiązywania problemów niektórzy naukowcy \cite{history-meta} . W 1945 Wydana została jego książka  \cite{polya} w której opisane były heurystyczne metody radzenia sobie z matematycznymi problemami. Polya twierdził, że problemy mogą być rozwiązywane przez ograniczony zbiór strategii, z których większość upraszcza problem. Mimo, że książka była skoncentrowana na matematycznych problemach, wiele propozycji sprawdzało się w rozwoju algorytmów optymalizacji.


%% The Appendices part is started with the command \appendix;
%% appendix sections are then done as normal sections
%% \appendix

%% \section{}
%% \label{}

%% References
%%
%% Following citation commands can be used in the body text:
%% Usage of \cite is as follows:
%%   \cite{key}          ==>>  [#]
%%   \cite[chap. 2]{key} ==>>  [#, chap. 2]
%%   \citet{key}         ==>>  Author [#]

%% References with bibTeX database:


%\bibliographystyle{iisthesis}
\bibliography{sample}


%% Authors are advised to submit their bibtex database files. They are
%% requested to list a bibtex style file in the manuscript if they do
%% not want to use model1-num-names.bst.

%% References without bibTeX database:

% \begin{thebibliography}{00}

%% \bibitem must have the following form:
%%   \bibitem{key}...
%%

% \bibitem{}

% \end{thebibliography}


\end{document}

%%
%% End of file `elsarticle-template-1-num.tex'.
